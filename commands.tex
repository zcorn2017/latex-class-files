
% ----- Packages --------------------------------------------------------------

\RequirePackage[outputdir=build, cache=false]{minted} % minted setup to include code
%\renewcommand{\MintedPygmentize}{/home/jpl/anaconda3/bin/pygmentize}
\setminted[python]{linenos, tabsize=2, breaklines}


\RequirePackage{geometry} % page margin
\RequirePackage{indentfirst}


% NOTE: Always put hyperef before gls
%\RequirePackage{glossaries}
%
%\makeglossaries
%\glstoctrue
%\loadglsentries{gls}
%\newcommand{\g}[1]{\gls{#1}}
%\newcommand{\gp}[1]{\glspl{#1}}



% Better fonts with accents
\RequirePackage[T1]{fontenc}

% Required for starred commands
\RequirePackage{suffix}

% Chemical Symbols
\RequirePackage{chemfig}

% Math symbols
\RequirePackage{amsmath}
\RequirePackage{amsfonts}
\RequirePackage{amsthm}
\RequirePackage{amssymb}
\RequirePackage{centernot}

% Nice lists
\RequirePackage{enumerate}
\RequirePackage{enumitem}

% Nice images, figures, and listings
\RequirePackage{grffile}
% \RequirePackage[all]{xy}
\RequirePackage{wrapfig}
\RequirePackage{fancyvrb}
\RequirePackage{listings}

% For better itemise using \1 \2 \3 \ldots
\RequirePackage{outlines}

\RequirePackage{dsfont} % \mathds
\RequirePackage{xcolor}

\definecolor{codegreen}{rgb}{0,0.6,0}
\definecolor{codegray}{rgb}{0.5,0.5,0.5}
\definecolor{codepurple}{rgb}{0.58,0,0.82}
\definecolor{backcolour}{rgb}{0.95,0.95,0.92}
\definecolor{eggray}{rgb}{0.4, 0.4, 0.35}

\lstdefinestyle{mystyle}{
	backgroundcolor=\color{backcolour},   
	commentstyle=\color{codegreen},
	keywordstyle=\color{magenta},
	numberstyle=\tiny\color{codegray},
	stringstyle=\color{codepurple},
	basicstyle=\ttfamily\footnotesize,
	breakatwhitespace=false,         
	breaklines=true,                 
	captionpos=b,                    
	keepspaces=true,                 
	numbers=left,                    
	numbersep=5pt,                  
	showspaces=false,                
	showstringspaces=false,
	showtabs=false,                  
	tabsize=2
}

\lstset{style=mystyle}


% Conditionals
\RequirePackage{ifthen}

% Header & Page Setup
\RequirePackage{fancyhdr}

% Links
\RequirePackage[colorlinks]{hyperref}
\hypersetup{
colorlinks,
linkcolor={red!50!black},
citecolor={blue!50!black},
urlcolor={blue!80!black}
}
\RequirePackage{cleveref}











% for pseudo code
\RequirePackage{algorithm}
\RequirePackage{algpseudocode}

\RequirePackage{chngcntr} % for counterwithin
\RequirePackage{calc} % get width of contents









\RequirePackage{tikz} % tikz
%\RequirePackage{tikzit} % with tikzit app


\RequirePackage{graphicx} % images
\setkeys{Gin}{width=0.6\linewidth}
\RequirePackage{wrapfig} % figure wrapped by texts
\RequirePackage{soul} % \ul
\RequirePackage{mathrsfs} % mathsrc
\RequirePackage{caption} % caption*

\RequirePackage{physics} % \dv and \pdv
\RequirePackage{gensymb} % \degree
\RequirePackage{linegoal} % \linegoal returns left width in a line
\RequirePackage{booktabs} % \toprule in tabular
\RequirePackage[d]{esvect} % \vv for vec





% SI units
\RequirePackage{siunitx}

% this function resolves error of repeated commands
%\newcommand*{\renameenviron}[1]{\expandafter\let\csname renamed-#1\expandafter\endcsname\csname #1\endcsname\expandafter\let\csname endrenamed-#1\expandafter\endcsname\csname end#1\endcsname\expandafter\let\csname #1\endcsname\relax\expandafter\let\csname end#1\endcsname\relax}


\newcommand*{\renameenviron}[1]{%
  \expandafter\let\csname exam-#1\expandafter\endcsname
      \csname #1\endcsname
  \expandafter\let\csname endexam-#1\expandafter\endcsname
      \csname end#1\endcsname
  \expandafter\let\csname #1\endcsname\relax
  \expandafter\let\csname end#1\endcsname\relax
}
\renameenviron{framed}
\renameenviron{shaded}
\renameenviron{leftbar}

% Thmbox
\RequirePackage{thmbox}
%\newtheorem[L]{theorem}{Theorem}[section]
%\newtheorem{theorem}{Theorem}[section]
%This is the example presented in the introduction but it has the additional parameter [section] that restarts the theorem counter at every new section.
%\newtheorem{corollary}{Corollary}[theorem]
%An environment called corollary is created, the counter of this new environment will be reset every time a new theorem environment is used.
%\newtheorem{lemma}[theorem]{Lemma}
%In this case, the even though a new environment called lemma is created, it will use the same counter as the theorem environment.
%\newtheorem*{remark}{Remark}
%\newtheorem{definition}{Definition}[section]
%\newtheorem*{remark}{Remark}
\newtheorem[L]{theorem}{Theorem}[section]
\newtheorem{corollary}[theorem]{Corollary}
\newtheorem[M]{definition}{Definition}[section]


% Text Boxes
\RequirePackage[]{mdframed}


% Fonts Settings
\RequirePackage{fontspec}
%\defaultfontfeatures{Mapping=tex-text,Scale=MatchLowercase}
%\setmainfont{OpenDyslexic3}
% \setmainfont{Atkinson Hyperlegible}

\setmainfont{Arial}
\setmonofont{Meslo LG L for Powerline}
%\setmonofont{JetBrains Mono}[Contextuals = Alternate,Ligatures = TeX]
%\crefname{questionCounter}{Good}{Note}


% Used for automatic embracing the text with brackets
\RequirePackage{xstring}

% use \centernot to negate any symbol
\RequirePackage{centernot}


\RequirePackage{cases}


% DEFINE ADDITIONAL MACROS
\newcommand\note{\textit{Note:\hspace{0.75cm}}}
\newcommand{\TODO}{\colorbox{red}{NOT DOING YET!}}
\newcommand{\question}{\textit{\frame{Question:}\hspace{0.75cm}}}
\newcommand{\claim}{\textit{\frame{Claim:}\hspace{0.75cm}}}
\newcommand{\recall}{\textit{\frame{Recall:}\hspace{0.75cm}}}

\newcommand{\mycell}[2][c]{\begin{tabular}[#1]{@{}c@{}}#2\end{tabular}} % used in tabular to force new line





% DEFINE MATH MACROS
% ----------------------------------------------------------------------------
% \newcommand{\qed}{\hfill\rule{2mm}{2mm}}
\DeclareMathOperator*{\argmax}{arg\,max}
\DeclareMathOperator*{\argmin}{arg\,min}
\DeclareMathOperator*{\LHS}{LHS}
\DeclareMathOperator*{\RHS}{RHS}
\newcommand{\floor}[1]{\lfloor #1 \rfloor}
\newcommand{\ceil}[1]{\lceil #1 \rceil}
\newcommand{\myrightarrow}[1]{\mathrel{\raisebox{-2pt}{$\xrightarrow{#1}$}}}

% Other commands
\renewcommand\st{\quad\text{s.t.}\quad}
\newcommand\E{\mathbb{E}} % expectation
\newcommand\F{\mathbb{F}} % fields
\newcommand\R{\mathbb{R}} % real
\newcommand\Q{\mathbb{Q}} % quadratic
\newcommand\Z{\mathbb{Z}} % integer
\newcommand\N{\mathbb{N}} % natural
\newcommand\bbC{\mathbb{C}} % complex
\newcommand\bbI{\mathbb{I}} % index function
\newcommand\bbS{\mathbb{S}}
\newcommand\calR{\mathcal{R}} % integrability
\newcommand\calF{\mathcal{F}} % family
\newcommand\calA{\mathcal{A}} % algebra
\newcommand\calB{\mathcal{B}} % Borel algebra
\newcommand\calM{\mathcal{M}}
\newcommand\bfi{\mathbf{i}}
\newcommand\bfj{\mathbf{j}}
\newcommand\bfk{\mathbf{k}}
\newcommand{\diam}{\operatorname{diam}}
\newcommand{\sumin}{\sum_{i=1}^n}
\newcommand{\sumiN}{\sum_{i=1}^N}
\newcommand{\sumiM}{\sum_{i=1}^M}
\newcommand{\sumjn}{\sum_{j=1}^n}
\newcommand{\sumjm}{\sum_{j=1}^m}
\newcommand{\sumil}{\sum_{i=1}^\ell}
\newcommand{\intab}{\int_{a}^{b}}

% Algebra
\newcommand{\GL}{\operatorname{GL}}
\newcommand{\SL}{\operatorname{SL}}
\newcommand{\Sym}{\operatorname{Sym}}
\newcommand{\lcm}{\operatorname{lcm}}
\newcommand{\sgn}{\operatorname{sgn}}
\newcommand{\im}{\operatorname{im}}
\renewcommand{\ker}{\operatorname{ker}}
\newcommand{\opNull}{\operatorname{null}}
\newcommand{\range}{\operatorname{range}}

% Analysis
\newcommand\simpleS{\boldsymbol{s}} % simple function
\newcommand{\ptwto}{\mathrel{\raisebox{-4pt}{$\xrightarrow{\text{pointwise}}$}}}
\newcommand{\unfto}{\mathrel{\raisebox{-4pt}{$\xrightarrow{\text{uniformly}}$}}}

% Machine learning
\newcommand\LossL{\mathcal{L}} % loss
\newcommand\RiskR{\mathcal{R}}
\newcommand\calN{\mathcal{N}} % normal distribution
\newcommand\calG{\mathcal{G}}
\newcommand\Var{\operatorname{var}} % variance
\newcommand\Cov{\operatorname{cov}} % covariance
\newcommand\Bias{\operatorname{Bias}} % bias
\newcommand\MSE{\operatorname{MSE}} % MSE loss
\newcommand{\MultiGaussianPDF}[1]{\frac{1}{\sqrt{(2\pi)^d|\Sigma_{#1}|}} \exp\left(-\frac12 (x-\mu_{#1})^T \Sigma_{#1}^{-1} (x-\mu_{#1})\right)}
%\newcommand{\algorithmSplit}[1]{\end{algorithmic} #1 \begin{algorithmic}[1]} % OLD TO BE DELETED
\newcommand{\idx}[1]{^{(#1)}} % superscript

% Linear Algebra you may change renewcommand to newcommand in Row
\newcommand{\Row}[1]{R_{#1}} % row
\newcommand{\x}[1]{x_{#1}} % x unknown
\newenvironment{amatrix}[1]{%
  \left[\begin{array}{@{}*{#1}{c}|c@{}}
}{%
  \end{array}\right]
}
% \newcommand{\ans}{\text{Answer: }} % Answer
% \newcommand{\rop}[1]{\stackrel{\substack{#1}}{\to}} % row operation
\newcommand{\rop}[1]{\xrightarrow{\substack{#1}}} % row operation

\newcommand{\tp}[1]{%
\StrLen{#1}[\mylength]%
\ifnum\mylength>1%
    (#1)^T%
\else%
    #1^T%
\fi} % transpose

\newcommand{\ivt}[1]{%
\StrLen{#1}[\mylength]%
\ifnum\mylength>1%
    (#1)^{-1}%
\else%
    #1^{-1}%
\fi} % invert

\newcommand{\hlight}[1]{%
  \ooalign{\hss\makebox[0pt]{\fcolorbox{green!30}{red!40}{$#1$}}\hss\cr\phantom{$#1$}}%
} % Highlight the element

\newcommand{\undertext}[2] {$\underbrace{\textrm{#1}}_{\textrm{#2}}$}
% -----------------------------------------------------------------------------




